\documentclass[journal]{IEEEtran}

% *** CITATION PACKAGES ***
%
%\usepackage{cite}
\usepackage{capt-of}%%To get the caption
\usepackage{gensymb}
\usepackage{graphicx} %package to manage images
\graphicspath{ {./images/} }
\usepackage{wrapfig}
\usepackage{xcolor}


\usepackage{amsmath}
\usepackage{amssymb}

\usepackage{lipsum}

\usepackage[style=ieee]{biblatex}
\DeclareLanguageMapping{english}{english-apa}
\addbibresource{references.bib}
\usepackage[justification=centering]{caption}

\usepackage{setspace}

\usepackage{hhline}


\usepackage{changepage} 

\usepackage{booktabs}
\usepackage{xcolor}

\usepackage{makecell}
\usepackage{graphicx,subcaption}
\usepackage{listings}
\renewcommand\theadfont{}
\DeclareMathOperator{\EX}{\mathbb{E}}% expected value
\newcommand{\cc}[1]{\texttt{#1}}


\usepackage{multicol} 

\raggedbottom

\begin{document}


\pagenumbering{gobble}
%\clearpage\mbox{} % adds and empty page
%\clearpage
\pagenumbering{arabic}
\setcounter{page}{1}

\title{Mini Project 3: Solve the Real World Problem Using TLC Data}

\author{ENGR-UH 4560 Selected Topics in  Information and Computational Systems Machine Learning, Fall 2019\\
\medskip
Barkin Simsek,~\IEEEmembership{bs3528@nyu.edu};
Nishant Aswani,~\IEEEmembership{nsa325@nyu.edu}}% <-this % stops a space


% The paper headers
\markboth{Simsek, Aswani, ENGR-UH 4560 Selected Topics in  Information and Computational Systems Machine Learning, Fall 2019}%
{}

% make the title area
\maketitle

% As a general rule, do not put math, special symbols or citations
% in the abstract or keywords.
\begin{abstract}
for the airplane
\end{abstract}

%%%%%%%%%%%%%%%%%%
%% Introduction %%
%%%%%%%%%%%%%%%%%%
\section{Introduction}
\subsection{Research Motivation}
{\color{blue} for the airplane}
\subsection{Research Objective}
{\color{blue} for the airplane}
\subsection{Scoping the Project}
\noindent Having plotted the pickup location heatmap, we decided to limit our exploration to Manhattan. Figure primarily because Manhattan showed to have the hottest pickup and dropoff locations. Lookin

%%%%%%%%%%%%%%%%%%%%%%%%%%%%%%%%%%%
%% Data Processing and Filtering %%
%%%%%%%%%%%%%%%%%%%%%%%%%%%%%%%%%%%
\section{Data Procurement, Processing, and Filtering}

\noindent Data was loaded into the workspace using the dask dataframe allowing for quicker access. As part of the \cc{read\_csv} function, the  \cc{tpep\_pickup\_datetime} and \cc{tpep\_dropoff\_datetime} columns were parsed as datetime formatted to allow for easier indexing and future data cleaning. Upon reviewing the dataframe, there were $1.846966e+07$ data points for 2017 and 2018 January. The challenge was then to identify methods for removing worthless datapoints and selecting a subset for training and testing. We decided against using the entire dataset because of limited computational resources. 

\subsection{Checking for Anomalies in TLC Data}

\noindent Once it was confirmed that there were no null values present in the 2017 and 2018 January dataset, we selected a {\color{red} 1\% fraction} of the dataset to work with. We then sorted and reindexed the data to obtain a dataframe of January datapoints.\\

\noindent To aid in finding unreasonable data, we turned to the Adopted Rules of the Taxi and Limousine Comission (TLC). The maximum trip duration is 12 hours: hence, we iterated through all columns to determine the trip duration by subtracting pickup time from dropoff time. Any duration greater than 12 hours was deemed to be {\color{blue} or in negative time}, was removed from the dataset. In each iteration we also calculated the speed and removed all results less than 0 or greater than 90.\\

\noindent For the remaining columns, such as \cc{total\_amount} and \cc{extra}, all rows with values less than zero were removed. 

\subsection{Weather Data}
\noindent Despite being relatively simple data, we ran into some trouble obtaining weather data for New York. As we had decided to work with Manhattan,  


%%%%%%%%%%%%%%%%%
%% Methodology %%
%%%%%%%%%%%%%%%%%
\section{Methodology}


%%%%%%%%%%%%%
%% Results %%
%%%%%%%%%%%%%
\section{Results}


%%%%%%%%%%%%%%%%%%%%%%%%%%%%%%%
%% Discussion and Conclusion %%
%%%%%%%%%%%%%%%%%%%%%%%%%%%%%%%
\section{Discussion and Conclusion}
\printbibliography


%%%%%%%%%%%%%%% figure example %%%%%%%%%%%%%%%%%%%%%%%%

% \begingroup
%     \centering
%     %width=\columnwidth
%     \includegraphics[width=\columnwidth]{images/scale-free.png}
%     \captionof{figure}{Scale-free network with n=500, e=1500. Size and color of each is proportional to that node's degree. Bigger and darker node have more connections. Smaller and lighter nodes have fewer connections.}
%     \label{fig:scalefree_network}
% \endgroup


%%%%%%%%%%%%%%% equation example %%%%%%%%%%%%%%%%%%%%%%%%

% \begin{equation}
%     \begin{split}
%         X &\texttt{\char`\~} Poisson(\lambda) \\
%         \\
%         P(X=k) &= \frac{\lambda^k e^{-\lambda}}{k!} \\
%         \\
%         \EX(k) &= e^{-\lambda}\sum_{k=1}^{\infty} \frac{\lambda^n}{(n-1)!}\\
%         \\
%         &= e^{-\lambda}\lambda\sum_{k=1}^{\infty} \frac{\lambda^{k-1}}{(k-1)!}\\
%         \\
%         &= e^{-\lambda}\lambda\sum_{n=0}^{\infty} \frac{\lambda^{n}}{(n)!}\\
%         \\
%         &= e^{-\lambda}\lambda e^{\lambda} = \lambda\\
%     \end{split}
%     \label{eq:mutual}
% \end{equation}

%%%%%%%%%%%%%%% table example %%%%%%%%%%%%%%%%%%%%%%%%

% \begingroup
%     \medskip
%     \centering
%     \def\arraystretch{1.5}
%         \begin{tabular}{lcc}
%             \toprule
%             & Random Network & Scale-Free Network \\
%             \midrule
%             Avg. of Avg. Degree    & 3.9988    &  4.0035   \\
%             Var. of Avg. Degree    & 7.6800$\times 10^{-7}$ & 3.1156$\times 10^{-6}$  \\
%             Avg. of Avg. Distance  & 5.5694 & 5.0935    \\
%             Var. of Avg. Distance  & 1.5507 & 0.5377   \\
%             \bottomrule
%         \end{tabular}
%     \captionof{table}{Statistics of fifty runs where \\ n = 5,000, e = 10,000}
%     \label{table:fifty_runs}
%     \medskip
% \endgroup

%%%%%%%%%%%%%%%%%%%%%%%%%%%%%%%%%%%%%%%%%%%%%
%%%%%%%%%%%%%%%%%%%%%%%%%%%%%%%%%%%%%%%%%%%%%
%%%%%%%%%%%%%%%%%%%%%%%%%%%%%%%%%%%%%%%%%%%%%

\newpage
\onecolumn
%%%%%%%%%%%%%%%%
%% Appendix %%
%%%%%%%%%%%%%%%%
\section{Appendix}

\end{document}